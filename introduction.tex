\chapter{Introduction and Motivation}


%application aspect of VR (training, rehabilitation, etc), Near field applications
There are several promising near field applications in \textit{Immersive Virtual Environments} (IVEs) that allow users to perform fine motor tasks in users' interaction space towards rehabilitation~\cite{DHH+13}, therapy~\cite{HAB+01}, and surgery training~\cite{S08}. State-of-the-art \textit{Virtual Reality} (VR) systems provide substantial advantages in facilitating repeatable and safe user interactions with the environment in simulating situations that are potentially dangerous, expensive or rare. Although a large amount of effort has been put into creating IVEs and user interaction metaphors to closely replicate the real world situations for the purpose of training and education, many studies have demonstrated that distance perception is distorted and erroneous in VEs~\cite{MD05,RW05} which has the potential to adversely affect task performance, training effectiveness, cause ocular-motor discomfort and simulator sickness. These kinds of distortions are problematic especially when VR is used to train skills involving fine motor activities geared towards transfer to the real world. In order to better understand how near field distance estimation operates in IVE on users, we took some measurements under different circumstances to illustrate that the depth mis-perception could be remedied via visuo-motor re-calibration and formulating methods to enhance spatial perception in IVE:

\begin{enumerate}
	\item \textit{Explore to what extent users are able to calibrate their depth judgments during visually guided actions in IVEs, when given congruent or dissonant visual and proprioceptive information while performing manual tasks, during 3D interactions in near-field VR.} One way to overcome the distortions in IVE is to allow users to interact with the VE in a natural manner utilizing 3D user interface metaphors that facilitate actions such as reaching and grasping for selection and manipulation \cite{ANL+12,LBB99,BKL+04}. However, feedback representing the users' actions in VR may consist of missing or maligned information in different visuo-motor sensory channels~\cite{CM03}. This may be due to technological limitations such as latency, tracker drift, registration errors, or intentional offsets between visual and proprioceptive information in the performance of near-field 3D interaction. These kinds of distortions could potentially alter one's perception of the environment. This fact has been shown by some research that our physical action is actually influenced by the presence of information from multiple sensory inputs such as visual, proprioceptive, auditory, and tactile channels~\cite{G66}. For instance, some research indicates that the visual and proprioceptive sensory channels are highly tied together and constantly calibrated based on sensory inputs from the real world~\cite{BP98}. Distortion in human spatial perception could potentially degrade training outcomes, experience and performance. Also, it has been shown that in the real world visuo-motor calibration rapidly alters one's actions to accommodate new circumstances \cite{BP98,BC12}. 
	
	\item \textit{Evaluate the effect of visual and haptic feedback on overcoming distance mis-perception by training the motor component of perception motor activities and illustrate that the users' notion of space could be affected by motor component.} Haptic feedback provides a sense of limb position and movement as well as an apprehension of object properties such as hardness, extent, orientation, weight and inertia \cite{PT92,pt98}. Haptic feedback and its role on precise perception has drawn more attention to it in VR \cite{bb+96}. 
	
	\item \textit{Characterize the physical reach motions for 3D interaction, and study how the reaches are different in real versus virtual worlds, and in the presence or absence of visuo-haptic information.} Understanding the properties of reach motion has applications not only to VR but also in areas such as animation, robotics, biomechanics, neuroscience, and ergonomics. Most often, human hand movements during spatial interaction to perform selection and manipulation tasks are executed via reach and are ballistic in nature \cite{ts55}. Rapid reaches to targets are characterized by a fast ballistic phase and then a much smaller and slower corrective phase. Past work has shown that the most accurate way to measure distance perception via rapid reaches is to use the end point of the fast ballistic phase \cite{BP98}. In such scenarios, users most often reach guided by visual information, and at other times using peripheral vision or no vision while reaching \cite{ss98}. Therefore, a comparative investigation of reach motions, examining the properties and characteristics of reaches, in the Real World (RW) as well as the IVE and in the presence or absence of vision and/or haptic feedback is important in understanding the characteristics of human motions under different interaction circumstances. As shown by many studies, the visual perceptual characteristics in VR are limited compared to the real world \cite{lk03}. It is not well understood how these limitations affect reaching motions during 3D interaction in the IVE.
	
	\item \textbf{Proposed Study}: \textit{Study the impact of animation fidelity on spatial perception in action space in IVE.} The interaction with the IVE could be through an immersive self-avatar which is life-size digital representation of the user. Existing of self-avatar requires some degrees of tracking and animation depending on accuracy of the rendered environment and the scenario. To generate high-fidelity immersive self-avatars in IVE one needs the higher level of real-time rending and animations which could be computationally expensive. However, there are some studies suggesting just the perception of a self-representation in IVE could increase the sense of presence \cite{LNW+03}. Other studies found the present of even a static self-avatar in the environment improved distance judgment via blind walking \cite{RIK+08,MCW+10}. In these previous studies, the self-avatar is mainly used to provide a sense of presence and was not involved in the calibration phase or the user's responses. Also, the perception of self-representation is only studied in far distance and less is known about the impact of animation fidelity on spatial perception in IVE in action space.
\end{enumerate}
















